\begin{frame}\frametitle{Biospheric Primary Production During an ENSO Transition} 
\begin{description}
    \item[Authors] Michael J. Behrenfield, James T. Randerson, Charles R. McClain, Gene C. Feldman, Sietse O. Los, Compton J. Tucker, Paul G. Falkowski, Christopher B. Field, Robert Frouin, Wayne E. Esaias, Dorota D. Kolber, Nathan H. Pollack
    \item[Objectives] To compare the simultaneous ocean and land net primary production (NPP) responses to El Nino to La Nina transitions utilizing SeaWiFS measurements of chlorophyll concentration ($C_{sat}$) in the oceans and Normalized Difference Vegetation Index (NDVI) on land.
    \item[Methods] Characterization of global, 4-km resolution, monthly SeaWiFS $C_{sat}$ and NDVI data sensed between Septermber 1997 and August 2000, an El Nino to La Nina transition period.
\end{description}
  \note[item]{Include notes and talking points here.}
  \note[item]{There can be more than one note.}
\end{frame}

\begin{frame}\frametitle{Biospheric Primary Production During an ENSO Transition} 
\begin{description}
    \item[Findings] The greatest increases in ocean NPP were found in regions most impacted by El Nino-Sourthern Oscillation (ENSO). The figure on the following slide shows\ldots
    \item[A)] Average NPP for the La Nina Austral summer of December 1998 to February 1999
    \item[B)] Average NPP for the La Nina Boreal summer of June to August 1999
    \item[C)] Transition from El Nino to La Nina conditions
    \item[D)] Changes in NPP between two La Nina Boreal summers
\end{description}
\end{frame}

\begin{frame}\frametitle{Biospheric Primary Production During an ENSO Transition} 
    \begin{center}
        \includegraphics[width=\textwidth,height=0.8\textheight,keepaspectratio]{npp}
    \end{center}
  \note[item]{Pretty picture}
\end{frame}
