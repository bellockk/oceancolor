\documentclass[xcolor=table,handout]{beamer}
\setbeamertemplate{navigation symbols}{}

\usepackage{pgfpages}
\usepackage{adjustbox}
\setbeameroption{show notes}
\setbeameroption{show notes on second screen=right}

\usepackage{booktabs}
\usepackage{beamerthemeshadow}
\setbeamertemplate{caption}[numbered]
\setbeamerfont{caption}{size=\tiny}

\begin{document}

\title{Journal Club Presentation: Ocean Color}  
\author[K. Bellock, L. Chen, D. Durrance, A. Yen]{Kenneth Bellock, Leshi Chen, Danielle Durrance, Andrew Yen}
\titlegraphic{%
    \includegraphics[height=.37\textheight]{15_037}

    \tiny \textbf{Source:} \url{https://www.nasa.gov/press/2015/march/new-nasa-mission-to-study-ocean-color-airborne-particles-and-clouds}
}
\date{\vspace*{-2.5em}\par April 4, 2018} 

\section{Introduction}

\begin{frame}
  \titlepage{}
  \note{Assumptions:}
  \note[item]{TODO:\@ List assumptions.}
\end{frame}

\begin{frame}\frametitle{Table of contents}
  \tableofcontents{}
  \note[item]{A road map is a great thing to have.}
  \note[item]{A joke or reference to current events in common culture would be great here if the audience appears receptive.}
\end{frame} 

\begin{frame}\frametitle{Introduction} 
Ocean Color refers to the multi-colored ocean map products created from the measurements of chlorophyll by global satellites. The satellites, equipped with multispectral scanners, collect water-leaving radiances and assess the absorptive properties of the Earth’s oceans.

Ocean color products are useful to gain insight of the physical and chemical processes occurring in Earth’s oceans. Chlorophyll concentration maps can be used for climate studies, measuring the salinity of oceans, estimating carbon exchange between oceans and atmosphere, and studying phytoplankton species.

  \note[item]{TODO:\@ Write Introduction}
\end{frame}


\section{Article Summaries}

\begin{frame}\frametitle{Biospheric Primary Production During an ENSO Transition} 

\begin{description}
    \item[Authors] Michael J. Behrenfield, James T. Randerson, Charles R. McClain, Gene C. Feldman, Sietse O. Los, Compton J. Tucker, Paul G. Falkowski, Christopher B. Field, Robert Frouin, Wayne E. Esaias, Dorota D. Kolber, Nathan H. Pollack
    \item[Objectives] TODO: Include objectives.
    \item[Methods] TODO: Include methods.
    \item[Findings] TODO: Include findings.
\end{description}

  \note[item]{Include notes and talking points here.}
  \note[item]{There can be more than one note.}
\end{frame}

\begin{frame}\frametitle{Biospheric Primary Production During an ENSO Transition} 
    \begin{center}
        \includegraphics[width=\textwidth,height=0.8\textheight,keepaspectratio]{npp}
    \end{center}
  \note[item]{Pretty picture}
\end{frame}

\begin{frame}\frametitle{Performance of the MODIS semi-analytical ocean color algorithm for chlorophyll-a} 
\begin{description}
    \item[Authors] K.L. Carder, F.R. Chen, J.P. Cannizzaro, J.W. Campbell, B.G. Mitchell
    \item[Objectives] TODO: Include objectives.
    \item[Methods] TODO: Include methods.
    \item[Findings] TODO: Include findings.
\end{description}

  \note[item]{Include notes and talking points here.}
  \note[item]{There can be more than one note.}
\end{frame}

\begin{frame}\frametitle{Performance of the MODIS semi-analytical ocean color algorithm for chlorophyll-a} 

  \note[item]{TODO:\@ Include a pretty picture.}
\end{frame}

\begin{frame}\frametitle{Decadal changes in global ocean chlorophyll} 
\begin{description}
    \item[Authors] Watson W. Gregg, Margarita E. Conkright
    \item[Objectives] The authors aim at finding decadal trends in global ocean chlorophyll between data obtained by CZCS (1979-1986) and those by SeaWiFS (1992– 2000).
    \item[Methods]
        \begin{itemize}
            \item Chlorophyll data from CZCS and SeaWiFS are combined for reanalysis at 1° spatial resolution. 
            \item To increase compatibility and to reduce residual errors, both archives are blended with in situ data. 
        \end{itemize}
\end{description}

  \note[item]{Include notes and talking points here.}
  \note[item]{There can be more than one note.}
\end{frame}

\begin{frame}\frametitle{Decadal changes in global ocean chlorophyll} 
\begin{description}
    \item[Findings] 
        \begin{itemize}
            \item There is large similarity in the global spatial distributions and seasonal variability between the two chlorophyll archives.  
            \item On average, the global ocean chlorophyll has decreased from the CZCS archive to the SeaWiFS by 6\%, and changes are mainly observed in summer and autumn.  
            \item Reductions in North Pacific and North Atlantics in summer are mainly caused by reduced wind stresses and warmer sea surface temperature (SST).  
            \item Regional meteorological events, such as PDP and ENSO have contributed to the changes in global ocean chlorophyll. 
        \end{itemize}
\end{description}
  \note[item]{Include a note if needed.}
\end{frame}

\begin{frame}\frametitle{Decadal changes in global ocean chlorophyll} 
    \hfill
    \includegraphics[width=\textwidth,height=0.8\textheight,keepaspectratio]{gregg1}
    \hfill
    \includegraphics[width=\textwidth,height=0.8\textheight,keepaspectratio]{gregg2}
    \hfill
  \note[item]{Pretty pictures}
\end{frame}

\begin{frame}\frametitle{Corrections to the Calibration of MODIS Aqua Ocean Color Bands Derived From SeaWiFS Data} 
\begin{description}
    \item[Authors] Gerhard Meister, Bryan A. Franz, Ewa J. Kwiatkowska, Charles R. McClain
    \item[Summary] \scriptsize A new calibration method was needed to correct a problem affecting the Moderate Resolution Imaging Spectroradiometer, MODIS Aqua’s, ability to provide accurate ocean color data. The system uses bands 8-14 to detect water leaving radiances from the Earth’s surface. The problem affected temporal information collected for wavelengths between 412-443 nm. Prior to the calibration issues, the MODIS Calibration and Support Team (MCST) used onboard calibrators and lunar irradiances to sufficiently calibrate the MODIS systems. Now, the calibration methods are based on the Ocean Biology Processing Group’s (OBPG) calibration solution for MODIS Terra, which experienced a similar problem. In this method, the MODIS system is cross-calibrated with SeaWiFS to recharacterize the data to correct for the temporal trend error. Now, ocean color data is made with SeaWiFS and MODIS Aqua data merged together. Each data set is processed on its own and then reconfigured.
\end{description}

  \note[item]{Include notes and talking points here.}
  \note[item]{There can be more than one note.}
\end{frame}

\begin{frame}\frametitle{Corrections to the Calibration of MODIS Aqua Ocean Color Bands Derived From SeaWiFS Data} 


\begin{columns}
\begin{column}{0.5\textwidth}
    \begin{center}
    \textbf{\scriptsize SEAWIFS SWATHS}
    \includegraphics[width=\textwidth,height=0.8\textheight,keepaspectratio]{N7Pw7r}

    \tiny \textbf{Source:} \url{http://epod.typepad.com/.a/6a0105371bb32c970b0115714d7c00970c-500wi}
    \end{center}
\end{column}
\begin{column}{0.5\textwidth}  %%<--- here
    \begin{center}
    \textbf{\scriptsize MODIS AQUA PREDICTED PATHS}
    \includegraphics[width=\textwidth,height=0.8\textheight,keepaspectratio]{kVvOUr}

    \tiny \textbf{Source:} \url{https://nsidc.org/sites/nsidc.org/files/aqua_tracks.20050215.gif}
    \end{center}
\end{column}
\end{columns}

\end{frame}


\section{Summary and Conclusion}

\begin{frame}\frametitle{Summary} 

  \note[item] These studies on ocean color discuss the evolution of chlorophyll-a measurements through advancements in remote sensing capabilities stemming from:
  \begin{itemize}
    \item Refining instrument calibration and assimilating datasets for long-term reanalysis.
    \item Better understanding of the effects of inter-annual and irregular climate phenomena on NPP and Phytoplankton absorption.
    \item Redesign of Chlorophyll-a algorithms for distinguishing between different absorptive ocean water constituents. 
  \end{itemize}
  All of these advancements have profound implications for a more clear understanding of the ocean's contribution to the Earth's carbon cycle.
\end{frame}

\begin{frame}\frametitle{Conclusion} 

  \note[item]{TODO:\@ Write Conclusion}
\end{frame}


\end{document}
